% !TEX TS-program = xelatex
% !TEX encoding = UTF-8 Unicode
% !Mode:: "TeX:UTF-8"

\documentclass{resume}
\usepackage{zh_CN-Adobefonts_external} % Simplified Chinese Support using external fonts (./fonts/zh_CN-Adobe/)
%\usepackage{zh_CN-Adobefonts_internal} % Simplified Chinese Support using system fonts
\usepackage{linespacing_fix} % disable extra space before next section
\usepackage{cite}

\begin{document}
\pagenumbering{gobble} % suppress displaying page number

\name{杨笑盈[前端]}

\basicInfo{
  \phone{\ 159-3567-7921 \ } 
  \faEnvelope{\ syzq2018@qq.com \ } 
  \faQq {\  876603678 \ } 
  \faRss \href{https://syzq.github.io/categories/}{\ 个人博客}}
  

 
\section{\faGraduationCap\  教育背景}
\datedsubsection{\textbf{山西农业大学}}{2014.09 -- 2018.07}
\textit{本科在读}\ 预计 2018 年 7 月毕业


\section{\faUser\ 个人经历}
\datedsubsection{\textbf{学习前端}}{2016.06 -- 2017.03}
% \role{实习}{经理: 高富帅}
\begin{itemize}
  \item 系统的自学了前端的基础知识
  \item 自学方式主要为 看书+技术文档+视频等
  
\end{itemize}

\datedsubsection{\textbf{仿百度百家号}}{2017.04}
\role{新闻聚合阅读平台,百度百家号 PC 端界面展示}{完整项目}
\begin{onehalfspacing}
演示地址:\faLink \ \href{https://syzq.github.io/baijiahao/}{https://syzq.github.io/baijiahao/}
\begin{itemize}
  \item 运用 HTML+CSS+Bootstrap 进行页面布局。
  \item 业务逻辑层主要使用 jQuery 和 underscore 的模板引擎。
  \item 新闻加载部分使用 Ajax 异步读取 JSON 数据。
\end{itemize}
\end{onehalfspacing}

\datedsubsection{\textbf{H5+C3动效页面}}{2017.06}
\role{基于 PC 端的动效页面,整个项目基本不借助第三方库}{练习项目}
\begin{onehalfspacing}
演示地址:\faLink \ \href{https://syzq.github.io/Demo/webJS/}{https://syzq.github.io/Demo/webJS/}
\begin{itemize}
  \item 由原生HTML(5)+CSS(3)+JavaScript搭建而成。
  \item 主要使用 HTML5 新的 API 和 CSS3 的2D\&3D变换以及 Animation 等动效。
\end{itemize}
\end{onehalfspacing}

\datedsubsection{\textbf{搭建个人博客}}{2017.09}
\role{托管在 GitHub上的静态博客}{学习总结}
\begin{onehalfspacing}
博客地址:\faLink \ \href{https://syzq.github.io/categories/}{https://syzq.github.io/categories/}
\begin{itemize}
  \item 基于 Git+Node.js+Hexo 搭建的快速、简洁且高效的博客框架。
  \item 博客内容主要由 Markdown 编辑并解析生成。
  \item 博客重新梳理和总结了一遍前端知识体系;博客内容包括但不限于前端知识。
\end{itemize}
\end{onehalfspacing}

% Reference Test
%\datedsubsection{\textbf{Paper Title\cite{zaharia2012resilient}}}{May. 2015}
%An xxx optimized for xxx\cite{verma2015large}
%\begin{itemize}
%  \item main contribution
%\end{itemize}

\section{\faCog\ IT 技能}
% increase linespacing [parsep=0.5ex]
\begin{itemize}[parsep=0.5ex]
  \item HTML \& CSS:能根据业务要求较好还原设计稿。
  \item JavaScript:有较好的原生JS知识功底,能一定程度脱离第三方库编程;阅读过《JavaScript高级程序设计》、《你不知道的JavaScript》等书籍。
  \item 前端框架:能较好利用 Bootstrap、jQuery 等提高开发效率;略微了解 React和 Vue。
  \item 后    端:了解 Node.js,能使用 Express 框架进行基本的后台环境搭建。会使用 MongoDB 存储数据。
  \item 其    他:会使用 npm 包管理工具以及 Webpack 配置基本的前端自动化开发环境。会使用 Git 进行版本管理;能使用一些基本的 Linux 命令。
\end{itemize}

% \section{\faHeartO\ 获奖情况}
% \datedline{\textit{第一名}, xxx 比赛}{2013 年6 月}
% \datedline{其他奖项}{2015}

\section{\faInfoCircle\ 其他}
% increase linespacing [parsep=0.5ex]
\begin{itemize}[parsep=0.5ex]
  \item 严谨踏实,有责任心,做事有计划,善于管理时间,执行力强,抗压能力较强。
  \item 协作能力强,有很强的团队意识,也有很强的独立性和适应能力。
  \item 乐于接受各种新事物,能较好的学习各种新知识,新技能。
\end{itemize}

%% Reference
%\newpage
%\bibliographystyle{IEEETran}
%\bibliography{mycite}
\end{document}
